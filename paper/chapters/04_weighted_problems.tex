\chapter{Weighted Sum Problems}\label{chapter:weighted}
In this chapter, we discuss the \textbf{NP-Hardness} of a few weighted problems. In weighted sum problems, 
there is a weighting function $w$ over a set $S$. The problems ask about whether the weight of this elements 
in $S$ satisfies a equality or inequality relationship. The weighted sum problems are also mathematical programming 
problems, which is another large discipline of \NPH\ problems. In Karp's paper, there were \XC, \SBS \footnote{What Karp presented was referred to as Knapsack, although
the definition is closer to the subset sum nowadays}, and \KN introduced. 
We present reductions from \XC\ to \SBS, from \SBS\ to \Part, \KN and \IP. 

\section{Subset sum}
We define the subset sum problem with a set and a weighting function. There are also a few alternative definitions using a multiset or a list without 
a weighting function, which is useful for our other reductions. More details follow in the number partition section. 
\problem{\SBS}{A finite set $S$, a weighting function $w$, and an integer $B$}{Is there a subset $S' \subseteq S$ s.t. 
\begin{equation}
    \sum_{x \in S'} w(x) = B \tag{1}
\end{equation}}

\subsection{Reduction Details}
We reduce the exact cover problem to the subset sum. We start with the exact cover problem over the sets of natural numbers. 
Given $(X, S)$ an exact cover instance over natural numbers, let $S$ be the set in the entry of the subset sum problem. Then we
define the weighting function $w$ and the sum $B$ by 
\begin{align*}
    w(s) = \sum_{x \in s} p^x , B = w(X)
\end{align*}
Where $p$ is a natural that is no less than $|S|$
The triple $(S, w, B)$ is then an input for the subset sum problem.
\begin{lemma}[Soundness]
    If $(X, S)$ is an instance of the exact cover. The reduced $(S, w, B)$ is then an instance of the subset sum problem. 
\end{lemma}
Apparently, the reduction is sound. Let the $S' \subseteq S$ be an exact cover of X. It then holds that
\begin{align*}
    \sum_{s \in S'} w(s) = \sum_{s \in S'} (\sum_{x \in s} p^x) = \sum_{x \in \bigcup S'} p^x = \sum_{x \in X} p^x = B
\end{align*}
Thus, $(S, w, B)$ is an instance of the subset sum problem. 
\begin{lemma}[Completeness]
    Let $(S, w, B)$ be reduced from $(X, S)$. If $(S, w, B)$ is a subset sum instance, $(X, S)$ has to be an exact cover instance. 
\end{lemma}
Obtain $S' \subseteq S$ for which \textbf{(1)} holds. We show the disjointness of $S'$ with contradiction. 
Assume that $S'$ is not disjoint. Then there exist $s_1, s_2 \in S'$ s.t. $s_1 \cap s_2 \not= \emptyset$. Let $x \in s_1 \cap s_2$ be arbitrary. Then 
there are two cases for the coefficient $c_x$ of $p^x$ in the polynomial $\sum_{x \in S'} w(x)$ of $p$. 
\begin{enumerate}
    \item $c_x \geq 2$. The corresponding coefficient $c_x'$ in the polynomial $w(X)$ is one. From \textbf{(1)} it is obvious that this case is not valid.
    \item $c_x = 1$ or $c_x = 0$. Though \textbf{(1)} is satisfied, there are still at least two $p^x$ in the polynomial $\sum_{x \in S'} w(x)$.
    Hence the number of $p^x$ in this polynomial is at least $p$. However, there are utmost $|S|$ elements in $S'$, meaning that there are utmost $|S|$ such $p^x$. From the 
    fact the $p > |S|$, it is not possible that the number of $p^x$ is greater equal $p$. As a result, this case is also not valid.
\end{enumerate}
In conclusion, the assumption is false and $S'$ is thus disjoint. Then, it follows directly from the disjointness that $S'$ covers $X$, otherwise \textbf{(1)}
is not satisfied. \\\\
This reduction is, however, limited to the exact cover over natural numbers. It is still necessary to generalize the reduction to any arbitrary type. For this reason,
we need to construct a mapping. 
\begin{lemma}
    Let $S$ be an arbitrary finite S. Then there exists a bijective function $f, S \rightarrow N$ s.t. the image of $f$ is empty when $S$ is empty, 
    and is $\{1, 2, ..., |S| - 1\}$ otherwise.
\end{lemma}
The proof for this lemma is trivial. With this approach, we can covert each exact cover problem to an exact cover problem over the natural numbers and then 
reduce it to the subset sum problem. 
\begin{lemma}[Polynomial Complexity]
    The construction of $(S, w, B)$ from $(X, S)$ is polynomial. 
\end{lemma}
When we map an arbitrary set into a natural number set as presented in Lemma 9, we have to iterate over the set $X$, which costs the complexity of $|X|$. 
Furthremore, we have to iterate over $S$ to construct $w$ and $B$, resulting in the complexity of $2|S|$. In total, it only costs $|X| + 2|S| \in$\bigO{|X| + |S|}, 
i.e. the linear complexity. 

\subsection{Implemantation Deatails}
\subsubsection{Reduction and Proof}
In implementation, we started with the construction as presented in Lemma 9. The function \textbf{map\_to\_nat} returns 
a function that maps $X$ to a corresponding set of natural numbers. Following this, we define the weighting function and 
reduction function. Similar to what happened with the exact hitting set, we also check if $X$ is finite and if $S$ is 
a collection before we perform a reduction, for these conditions are a requirement under our definition. 
\Snippet{subset-sum-def}
Apart from the definition, we need a lemma that converts the sum to a polynomial. To guarantee that $p^k$ is unique for an arbitrary $k$, 
we require that $p \geq 2$ as in the lemma \textbf{weight\_eq\_poly}.
\Snippet{subset-sum-aux1} 
The main part of the proof is implemented as discussed in the reduction details. A problem occurred when showing the completeness of the reduction. 
Besides the proof presented in the implementation details, it is necessary to show that the representation of a natural number in the form 
of the polynomials with the base $p$ is unique. To achieve this, we imported the Archive of Formal Proofs entry DigitInBase and applied the theorem 
\textbf{seq\_uniqueness}.
\Snippet{subset-sum-aux2}
Then, we can show that the reduction is correct.
\Snippet{subset-sum-aux3}


\subsubsection{Polynomial Complexity}
The implementation of the proof for polynomial complexity is identical to what is introduced in reductio details. Fortunately, 
there is not additional step to be stated, for all of the proof is automated. 
\Snippet{subset-sum-poly}

\section{Subset sum in list and number partition}
The next problem that we want to reduce to is partition.
\problem{\Part}{A finite sequence $as$ of natural numbers.}{Is there a sub-sequence $as' \subset s$ s.t. 
\begin{align*} 
    \sum_{x \in as'} x = \sum_{x \in  as - as'} x \tag{2}
\end{align*}
}
For the modeling of the sequence, we use the list. Although it is possible 
to define the partition problem using the set and weighting function as in the subset
sum problem, choose the presented definition for two reasons.
\begin{enumerate}
    \item Showing that the definition of the problem is not an important factor 
    in reduction, i.e. both definition are valid and reducible under Isabelle.
    \item Staying consistent with the available Archive of Formal Proof instance 
    \textit{Hardness of Lattice Problems}, which is also the purpose of this work,
    i.e. providing theoretical bases for a few other verification projects.
\end{enumerate} 
Thus, we need to perform the reduction from the list version of the subset sum problem. 
We give the definition of the subset sum problem using a sequence. 
\problem{\textbf{Subset Sum in Sequence}}{
    A finite sequence $as$ of natural numbers, a natural number $s$}{
    Is there a sub-sequence $as' \subset as$ s.t. 
    \begin{align*}
        \sum_{x \in as'} x = B \tag{3}
    \end{align*}
}
Given the subset sum instance $(S, w, B)$, it is obvious that we can obtain a sequence $as$ by
converting $S$ into a sequence and map the sequence with the weighting function $w$. Let $s = B$.
The resulting pair $(as, s)$ is then an instance of the subset sum problem in sequence representation. 
Then we reduce $(as, s)$ to a partition instance $bs$.
\begin{align*}
    bs = (1 - s + \sum_{x \in as} x) \# (s + 1) \# as
\end{align*}
\begin{lemma}[Soundness]
    If there exists an $as'$ s.t. the equation \textbf{(3)} holds for $(as, B)$, \textbf{(2)} should hold for the reduced $bs$.
\end{lemma}
We construct a $bs'$ from $as'$ by 
\begin{align*}
    bs' &= (1 - s + \sum_{x \in as} x) \# as' \\
    bs - bs' &=  (s + 1) \# (as - as')
\end{align*}
Where the sums of the sequences satisfy the equation 
\begin{align*}
    \sum_{x \in bs'} x = (1 - s + \sum_{x \in as} x) + \sum_{x \in as'} x
    = (1 - s + \sum_{x \in as} x) + s = (s + 1) + (\sum_{x \in as} x - s)
    = \sum_{x \in bs - bs'}
\end{align*}
Thus, the reduction is sound.
\begin{lemma}[Completeness]
    Let $bs$ be reduced from $(as, B)$. If there exists a $bs'$ s.t. \textbf{(2)} holds for $bs$, 
    \textbf{(3)} should then hold for $(as, B)$.
\end{lemma}
It holds that
\begin{align*}
    (1 - s + \sum_{x \in as} x) + (s + 1) = \sum_{x \in as} x + 2 > \sum_{x \in as} x
\end{align*}
As a result, $(1 - s + \sum_{x \in as} x)$ and $s + 1$ are not supposed be simultaneously existent in $bs'$. 
After separating the first two elements of bs into different subsequences, the $as'$ is constructed 
by obtaining the tail of $bs'$, with which the completeness is proven. 
\begin{lemma}[Polynomial Complexity]
    The reduction from the subset sum to partition is polynomial.
\end{lemma}
The conversion between the definitions of the subset sum problem costs linear complexity w.r.t. the cardinality of the set,
for it it necessary to iterate the set $S$ and map the sequence with the weighting function. More specifically it costs 
the complexity of $|S| + 1$. Furthermore, we iterate the sequence similary when reducing the subset sum to partition, costing
the complexity of $|as| + 2$. In total, the complexity is \bigO{(|S| + 1) + (|as| + 2)} $\in$ \bigO{|S|} because of $|as| = |S|$.

\subsection{Implemantation Details of Subset Sum in Sequence}
\subsubsection*{Reduction and Proof}
Although the reduction is more straightforward compared to the previously introduced ones, the implementation is even lengthier. 
The reason is that conversion of the set to a list also is a reduction, which we have to verify. The intermediate step is then
defined by 
\Snippet{ss-indices-def}
Apparently, to map $S$ to an set of integers from 1 to $|S|$, we need to apply \textbf{Lemma 9} again. Additionally, we have to 
check if $S$ is finite, for finiteness is a requirement of the reduction.
Thus, the reduction and the proof 
is given by
\Snippet{ss-indices-red}
Then, it suffices to convert the set into a list and perform the map, in which we used the function 
\textbf{sorted\_list\_of\_set}, converting a set of ordered type to a sorted list. Similarly, we check if 
$S$ is finite and if $S$ is of form $\{1, 2, .. , |S|\}$ as a requirement. The proof for the correctness is then mostly straightforward
after unfolding the necessary definitions and using a few available lemmas in the list library, such as \textbf{nth\_equalityI} etc. 
\Snippet{ss-list}

\subsubsection*{Polynomial Complexity}
Similar to the reduction from exact cover to subset sum, the proof for the polynomial complexity for two reductions 
is rather trivial. 
\Snippet{ss-indices-poly}
\Snippet{ss-list-poly}

\subsection{Implemantation Details of Partition}
Instead of the original definition, where the sum a sub-sequence is equal to another, we use a different definition in the implementation,
where the twice of the sum of a sub-sequence is equal to to the sum of the sequence. We have also shown that this definition is equivalent 
to the original definition, i.e. \textbf{part\_alter} in implementation. 
\Snippet{part-def}
The reason was initially for the convenience of the proof. 
In the original definition, it is necessary to consider the sum of the sub-sequence $(as - as')$ when showing the soundness lemma. This is, unfortunately, 
rather complex under our definition, for we have to flip the list $xs$, the zero-one list that is used for multiplication. For this 
flipping operation, we have show the lemma \textbf{sum\_\binary\_part}. If we use the new definition, this is avoidable.
\Snippet{part-aux1}
However, when showing the completeness lemma, we found out that we have to show the same statement for the new definition, too. 
Thus, it is not a absolutely better definition. \\\\
The proof of the polynomial complexity is also trivial. 
\begin{quote}
    Should I include the codes without much description, or leave them unincluded? 
    Possible description: choice of the complexity and what is done in each step, but would be too lengthy.
\end{quote}

\section{Knapsack and Zero-One Integer Programming}
Knapsack and zero-one integer programming are another two classical weighted sum problems. While subset sum was referred 
to as knapsack in Karp's paper, its definition is nowadays different. Additionally, zero-one integer programming 
was originally reduced from \SAT. Nevertheless, there exists a trivial reduction from subset sum to both of the problems. Thus, 
we include them in this chapter and present a reduction for them each. 
Since the reduction is short and trivial, the implementation does not have 
many meaningful details and is thus omitted. 

\subsection{Knapsack}
\problem{\KN}{A finite set S, a weighting function $w$, a limiting function $b$, a upperbound $W$, a lowerbound $B$}{
    Is there a subset $S' \subseteq S$ s.t. 
    \begin{align*}
        \sum_{x \in S'} w(x) &\leq W \\ 
        \sum_{x \in S'} b(x) &\geq B
    \end{align*}
}
We reduce $\SS$ to $\KN$. With $(S, w, B)$ as an instance of the subset sum, $(S, w, w, B, B)$
is then an instance of knapsack with 
\begin{align*}
    \sum_{x \in S'} w(x) &= W \\ 
    \sum_{x \in S'} b(x) &= B
\end{align*}
where $b = w$ and $W = B$. Apparently, the reduction is constant, for no iteration is necessary. 

\subsection{Zero-one Integer Programming}
\problem{\IP}{A finite set $X$ of pairs $(x, b)$, where $x$ is an $m$-tuple of integers and $b$
is an integer, an $m$-tuple $x$ and an integer $B$}{
    Is there an $m$-tuple $y$ of integers s.t.
    \begin{align*}
        x^T \cdot y &\leq b \\ 
        c^T \cdot y &\geq B, \forall (x, b) \in X
    \end{align*}
}
Although most researchers tend to use matrix for the 
definition of the zero-one integer programming, we follow the definition from the intractability book(to cite), because 
it is convenient for our definition and consequently requires less effort. Given an instance of subset sum problem in sequence,
(as, s), let 
\begin{align*}
    X = \{(as, s)\}, c = as, B = s
\end{align*} 
$(X, c, B)$ is then an instance of the zero-one integer programming problem,
for there exists an $xs$ s.t. $xs^T \cdot as = s$. Since there is no iteration, the reduction is contant, too.

