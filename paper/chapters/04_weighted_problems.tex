\chapter{Weighted Sum Problems}\label{chapter:weighted}
In this chapter, we discuss the \textbf{NP-Hardness} of a few weighted problems. In weighted sum problems, 
there is a weighting function $w$ over a set $S$. The problems ask about whether the weight of this elements 
in $S$ satisfies a equality or inequality relationship. The weighted sum problems are also mathematical programming 
problems, which is another large discipline of \NPH\ problems. In Karp's paper, there were \XC, \SBS \footnote{What Karp presented was referred to as Knapsack, although
the definition is closer to the subset sum nowadays}, and \KN introduced. 
We present reductions from \XC\ to \SBS, from \SBS\ to \Part, \KN and \IP. 

\section{Subset sum}
We define the subset sum problem with a set and a weighting function. There are also a few alternative definitions using a multiset or a list without 
a weighting function, which is useful for our other reductions. More details follow in the number partition section. 
\problem{\SBS}{A finite set $S$, a weighting function $w$, and an integer $B$}{Is there a subset $S' \subseteq S$ s.t. 
\begin{equation}
    \sum_{x \in S'} w(x) = B \tag{1}
\end{equation}}

\subsection{Reduction Details}
We reduce the exact cover problem to the subset sum. We start with the exact cover problem over the sets of natural numbers. 
Given $(X, S)$ an exact cover instance over natural numbers, let $S$ be the set in the entry of the subset sum problem. Then we
define the weighting function $w$ and the sum $B$ by 
\begin{align*}
    w(s) = \sum_{x \in s} p^x , B = w(X)
\end{align*}
Where $p$ is a natural that is no less than $|S|$
The triple $(S, w, B)$ is then an input for the subset sum problem.
\begin{lemma}[Soundness]
    If $(X, S)$ is an instance of the exact cover. The reduced $(S, w, B)$ is then an instance of the subset sum problem. 
\end{lemma}
Apparently, the reduction is sound. Let the $S' \subseteq S$ be an exact cover of X. It then holds that
\begin{align*}
    \sum_{s \in S'} w(s) = \sum_{s \in S'} (\sum_{x \in s} p^x) = \sum_{x \in \bigcup S'} p^x = \sum_{x \in X} p^x = B
\end{align*}
Thus, $(S, w, B)$ is an instance of the subset sum problem. 
\begin{lemma}[Completeness]
    Let $(S, w, B)$ be reduced from $(X, S)$. If $(S, w, B)$ is a subset sum instance, $(X, S)$ has to be an exact cover instance. 
\end{lemma}
Obtain $S' \subseteq S$ for which \textbf{(1)} holds. We show the disjointness of $S'$ with contradiction. 
Assume that $S'$ is not disjoint. Then there exist $s_1, s_2 \in S'$ s.t. $s_1 \cap s_2 \not= \emptyset$. Let $x \in s_1 \cap s_2$ be arbitrary. Then 
there are two cases for the coefficient $c_x$ of $p^x$ in the polynomial $\sum_{x \in S'} w(x)$ of $p$. 
\begin{enumerate}
    \item $c_x \geq 2$. The corresponding coefficient $c_x'$ in the polynomial $w(X)$ is one. From \textbf{(1)} it is obvious that this case is not valid.
    \item $c_x = 1$ or $c_x = 0$. Though \textbf{(1)} is satisfied, there are still at least two $p^x$ in the polynomial $\sum_{x \in S'} w(x)$.
    Hence the number of $p^x$ in this polynomial is at least $p$. However, there are utmost $|S|$ elements in $S'$, meaning that there are utmost $|S|$ such $p^x$. From the 
    fact the $p > |S|$, it is not possible that the number of $p^x$ is greater equal $p$. As a result, this case is also not valid.
\end{enumerate}
In conclusion, the assumption is false and $S'$ is thus disjoint. Then, it follows directly from the disjointness that $S'$ covers $X$, otherwise \textbf{(1)}
is not satisfied. \\\\
This reduction is, however, limited to the exact cover over natural numbers. It is still necessary to generalize the reduction to any arbitrary type. For this reason,
we need to construct a mapping. 
\begin{lemma}
    Let $S$ be an arbitrary finite S. Then there exists a bijective function $f, S \rightarrow N$ s.t. the image of $f$ is empty when $S$ is empty, 
    and is $\{1, 2, ..., |S| - 1\}$ otherwise.
\end{lemma}
The proof of this lemma is trivial. With this approach, we can covert each exact cover problem to an exact cover problem over the natural numbers and then 
reduce it to the subset sum problem. 
\begin{lemma}[Polynomial Complexity]
    The construction of $(S, w, B)$ from $(X, S)$ is polynomial. 
\end{lemma}
When we map an arbitrary set into a natural number set as presented in Lemma 9, we have to iterate over the set $X$, which costs the complexity of $|X|$. 
Furthremore, we have to iterate over $S$ to construct $w$ and $B$, resulting in the complexity of $2|S|$. In total, it only costs $|X| + 2|S| \in$\bigO{|X| + |S|}, 
i.e. the linear complexity. 

\subsection{Implemantation Deatails}
\subsubsection{Reduction and Proof}
In implementation, we started with the construction as presented in Lemma 9. The function \textbf{map\_to\_nat} returns 
a function that maps $X$ to a corresponding set of natural numbers. Following this, we define the weighting function and 
reduction function. Similar to what happened with the exact hitting set, we also check if $X$ is finite and if $S$ is 
a collection before we perform a reduction, for these conditions are a requirement under our definition. 
\Snippet{subset-sum-def}
Apart from the definition, we need a lemma that converts the sum to a polynomial. To guarantee that $p^k$ is unique for an arbitrary $k$, 
we require that $p \geq 2$ as in the lemma \textbf{weight\_eq\_poly}.
\Snippet{subset-sum-aux1} 
The main part of the proof is implemented as discussed in the reduction details. A problem occurred when showing the completeless of the reduction. 
Besides the proof presented in the implementation details, it is necessary to show that the representation of a natural number in the form 
of the polynomials with the base $p$ is unique. To achieve this, we imported the Archive of Formal Proofs entry DigitInBase and applied the theorem 
\textbf{seq\_uniqueness}.
\Snippet{subset-sum-aux2}
Then, we can show that the reduction is correct.
\Snippet{subset-sum-aux3}


\subsubsection{Polynomial Complexity}
The implementation of the proof of polynomial complexity is identical to what is introduced in reductio details. Fortunately, 
there is not additional step to be stated, for all of the proof is automated. 
\Snippet{subset-sum-poly}

\section{Subset sum in list and number partition}
The next problem that we want to reduce to is partition.
\problem{\Part}{A finite sequence $s$ of natural numbers.}{Is there a sub-sequence $s' \subset s$ s.t. 
\begin{align*} 
    \sum_{x \in s'} x = \sum_{x \in  s - s'} x
\end{align*}
}
