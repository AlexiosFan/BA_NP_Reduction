\chapter{Set Covering Problems}\label{chapter:covering}
In this chapter, we discuss about the NP-hardness of a few set covering problems. 
Covering problems ask whether a certain combinatorical structure $A$ covers another structure $B$. 
Alternatively, it may also ask for the minimal size of $A$ to cover $B$. 
We focus on a subclass of covering problems, the exact covering problem. In this subclass, 
$A$ covers $B$ exactly, i.e.\ no element in $B$ is covered twice in $A$. 
In Karp's paper in 1972 \cite{karp2010reducibility}, the following covering problems were included: Exact Cover, Exact Hitting Set,
3-Dimensional Matching, and Steiner Tree. In our implementation, we reduced Satisfiability to Exact Cover, and then reduced Exact Cover to Exact Hitting Set. 

\section{Exact Cover}
The Exact Cover problem is a special case of the Set Cover. 
Exact Cover problems not only look for a cover, 
but also require the existence of covered elements to be unique. 
\problem{Exact Cover}{A set $X$ and a collection $S$ of subsets of $X$.}{Is there a disjoint subset $S'$ of $S$ s.t.\ 
each element in $X$ is contained in one of the elements of $S'$?
\label{eq:1} }
\begin{myalign}
    \textbf{Exact Cover} &:= \{(X, S) | \bigcup S \subseteq X \land \exists S' \subseteq S.\\
    & \forall x \in X. \exists s \in S'. x \in s\\
    & \land\forall s\ t \in S'. s \neq t \longrightarrow s \cap t = \emptyset
    \}
\end{myalign}
We call $(X, S)$ an instance of Exact Cover and $S'$ an exact cover of $X$. In other words, $S'$ covers $X$ 
exactly.


\subsection{Choice of Reduction}
Since the Exact Cover problem is a fundamental NP-hard problem, there are many different reductions found by different researchers. 
Karp's reduction\cite{karp2010reducibility} is based on the  Chromatic Number problem. 
Although the Chromatic Number problem was formalised in Karp's 21 project, we did not choose this reduction
because of the complexity of the graph traversal and the differences between Karp's definition and the available Isabelle's definition. 
In the available definition of Chromatic Number, the Chromatic Number $k$ is limited to be at least three. Since 
the Chromatic Number is in \textbf{\textit{P}} in the case $k = 2$, the alternative definition is correct. However, this raises the difficulty of performing 
a reduction, for it is necessary to consider a special case for $k = 2$.\\\\ 
On the contrary, there is an easy reduction from Satisfiability to Exact Cover. This reduction does not involve graph traversal.
The only technical barrier is the typeless set. While Isabelle only supports typed sets, we resolve this problem
by creating an encapsulation type. More details follow in the section \ref*{sec:sat-imp}. 

\subsection{Reduction Details}
Now we present the reduction from Satisfiability to Exact Cover. 
Before defining the reduction function, we first state the notations that we use for this part. 
Given a propositional logical formula $F$, $vars\; F$ denotes the set of variables in $F$. 
An assignment $\sigma$ is a function that maps a variable to a true-false value. 
With $\top$ as true and $\bot$ as false, $\sigma(x_i) = \top$ reads $x_1$ is true under the assignment $\sigma$. \\\\
The assignment is also defined over the formula $F$. 
$F$ is satisfiable under $\sigma$ if there is at least one variable that is true under $\sigma$ in each clause of $F$. 
This is represented by $\sigma \models F$. \\\\
Additionally, we index the variables and the clauses and use the following notations.
\begin{enumerate}
    \item $x_i$ denotes the $i$-th variable in the formula with $x_i \in vars\; F$
    \item $c_i$ denotes the $i$-th clause in the formula with $c_i \in F$
    \item $p_{ij}$ denotes the $j$-th position/literal in the $i$-th clause with $p_{ij} \in c_i$
\end{enumerate} 
Then we construct a set $X$ and which contains all three different kinds of objects. 
\begin{align*}
    X = vars\; F \cup F \cup \bigcup_{c_i \in F} c_i
\end{align*}
Furthermore, we construct $S$, a collection of subsets of $X$. We determine the following subsets
\begin{enumerate}
    \item $\{p_{ij}\}$. The unary set of positions
    \item $\{c_i, p_{ij}\}$. The binary set of a clause and a position inside the clause.
    \item $pos(x_i) := \{x_i\} \cup \{p_{ab} | p_{ab} = x_i\}$. The set of a variable with its positive occurrences as positions.
    \item $neg(x_i):= \{x_i\} \cup \{p_{ab} | p_{ab} = \neg x_i\}$. The set of a variable with  its negative occurrences as positions.
\end{enumerate}
$S$ contains all of the four types of subsets.
\begin{align*}
    S =& \{{p_{ij}} | p_{ij} \in c_i, c_i \in F \} 
    \cup \{\{c_i, p_{ij}\} | p_{ij} \in c_i, c_i \in F \} \\
    &\cup \{\{x_i\} \cup \{p_{ij} | p_{ij} \in c_i, c_i \in F\} | x_i \in vars\; F, x_i = p_{ij}\}\\
    &\cup \{\{x_i\} \cup \{p_{ij} | p_{ij} \in c_i, c_i \in F\} | x_i \in vars\; F, \neg x_i = p_{ij}\}
\end{align*}
\begin{definition}[Reduction SAT to XC]
    Given an instance of Satisfiability $F$, 
    it is reduced to an instance of Exact Cover $(X, S)$ as presented above.
\end{definition}
After definiting the reduction function,
we also show that the reduction is correct and in polynomial-time.
\begin{lemma}[Soundess]
    Let $F$ be a satisfiable propositional logical formula. The pair $(X, S)$ is then an instance of the exact cover.
\end{lemma}
\begin{proof}
    Let $\sigma \models F$ be a valid assignment. We construct an exact cover $S' \subseteq S$ of $X$ in the following steps.
\begin{enumerate}
    \item For each $x_i \in vars\; F$, $pos(x_i)$ is included in $S'$ when $\sigma(pos(x_i)) \equiv \bot$. Otherwise we insert $neg(x_i)$ into $S'$.
    This steps covers all variables and all positions that are false under $\sigma$.
    \item For each $c_i \in F$, we choose the minimal $j$ with $\sigma(p_{ij}) \equiv \top$, and insert $\{c_i, p_{ij}\}$ into $S'$.
    This steps covers all clauses and one position that is true under $\sigma$ in each clause.
    \item For each $p_{ij} \in c_i$, if $\sigma(p_{ij}) \equiv \top$ and $\{c_i, p_{ij}\}$ is not in $S'$, the unary set $\{p_{ij}\}$ is included. 
    This steps covers all positions that are true under $\sigma$ and are not covered in the previous step.
\end{enumerate}
Now it suffices to show that $S'$ is disjoint in our construction.
Obviously, each position in $pos(x_i)$ and $neg(x_i)$ is false under the assignment $\sigma$, 
while the positions in the other sets are all true. By design, the positions in the second and the third steps never duplicate. 
Thus, no same position occurs in two different sets in the collection $S'$. 
Furthermore, since each clause and variable is unique in the construction, no two sets contain the same clause or the same variable.
Hence $S'$ is disjoint.
\end{proof}
For the correctness, we have to show the completeness, too.
\begin{lemma}[Completeness]
    Let $(X, S)$ be reduced from $F$. If $(X, S)$ is an instance of the Exact Cover, $F$ has to be satisfiable.
\end{lemma}
\begin{proof}
    Given an instance of Exact Cover $(X, S)$ reduced from $F$, 
    it is easy to reconstruct the model $\sigma$ with the same approach as in the proof of the soundness. 
    We iterate over the variables. For each variable $x_i$, if $pos(x_i) \in S'$, we set $\sigma(x_i) = \bot$. 
    Otherwise, $neg(x_i)$ has to be in $S'$, for $x_i$ is covered exactly. In this case, we set $\sigma(x_i) = \top$. \\\\
    Now, we show that this assignment is valid. For each clause $c_i$, we obtain the unique set $\{c_i, p_{ij}\} \in S'$.
    There are two cases for $p_{ij}$.
    \begin{enumerate}
        \item $p_{ij} = x_k$. $neg(x_k)$ is then covered in $S'$, resulting in $\sigma(p_{ij}) = \sigma(x_k) = \top$.
        \item $p_{ij} = \neg x_k$. $pos(x_k)$ is then covered in $S$. resulting in $\sigma(p_{ij}) = \sigma(\neg x_k) = \neg \sigma(x_k) = \top$
    \end{enumerate}
    Thus, for each clause, there is at least one position that is true under $\sigma$. 
    Consequently, $\sigma$ is a valid assignment of $F$.
\end{proof}
In the end, we present a complexity analysis.
\begin{lemma}[Polynomial Complexity]
    The construction of $(X, S)$ from $F$ can be computed within polynomial time. 
\end{lemma}
\begin{proof}
    In the reduction, we have to iterate all of the variables, the clauses and the positions. 
    Thus, we have to find a polynomial bound with regards to all of the three metrics. Let $n, m$ and $k$ denote 
    the number of variables, clauses and postitions. To obtain the three metrics, we have to iterate 
    all of the clauses, resulting the complexity of $\bigO{m}$. With these iterations, we can construct the set $X$,
    indicating a linear complexity for the construction of $X$. \\\\
    Now the interesting part is the collection $S$. For each type of subsets, we present a polynomial bound 
\begin{enumerate}
    \item $\{p_{ij}\}$. It is sufficient to iterate the positions, requiring the complexity of $k$.
    \item $\{c_i, p_{ij}\}$. Each clause is iterated for $|c_i|$ times. Since $|c_i|$ is a constant, there is a $c \in \mathbb{N}$ 
    s.t. $|c_i| \leq c$. Then, the complexity is above bounded by $c \cdot m$.
    \item $pos(x)$ and $neg(x)$. For each variable $x$, it is required to iterate all of the positions, which produces the complexity 
    of $2 \cdot nk$ in total.
\end{enumerate}
Thus, the construction of S costs the polynomial complexity of $k + cm + 2nk \in \mathcal{O}(nk + m)$. 
With the linear complexity of the construction of $X$, we conclude that the reduction has the quadratic complexity.
\end{proof}
In the end, we summarize the lemmas to obtain the central theorem.
\begin{theorem}
    The presented reduction is a polynomial-time reduction from Satisfiability to Exact Cover.
\end{theorem}

\subsection{Implementation Details}
\label{sec:sat-imp}
\subsubsection{Choice of Definitions}
A definition of Exact Cover, as shown in \hyperref[figure:3]{Figure 3.1}, states the disjointness with nested quantifiers.
It is an implementation of the mathematical definition in \hyperref[eq:1]{Definition 6}.
\begin{figure}[h!]
    \Snippet{exact-cover-original}
    \caption{A first definition of Exact Cover}
    \label{figure:3}
\end{figure}
Three different conditions have be satisfied for an instance of exact cover.
The first condition $finite\ X$ checks if $X$ is a finite set, while the second 
condition $\bigcup S \subseteq X$ limits $S$ to be a collection of $X$. The third 
condition states about the existence of the exact cover. Although this definition gives us a direct 
view of the problem, it is lengthy for the further implementation. Hence we choose to use 
an alternative definition in \hyperref[figure:4]{Figure 3.2} with the help of the dependency \textbf{HOL-Library.Disjoint\_Sets}.
\begin{figure}[h!]
    \Snippet{exact-cover-def}
    \caption{A second definition of Exact Cover}
    \label{figure:4}
\end{figure}
With the pre-defined predicate \textbf{disjoint}, we managed to simplify the definition. Moreover, 
we also benefitted from reusing the existing lemmas in the relating dependency.

\subsubsection{Encapsulation Type for the Construction}
Intuitively, the variables, positions and clauses can be represented by tuple of indices. However, it is not easy 
to unify the representation such that they are of the same type. For example, a unary set of position is representable with
one single index, whereas a binary set of a clause and a position needs at least two indices. 
On the contrary, an encapsulation type is exempted from the tedious handling of tuples.
Thus, we implement this encapsulation type \textbf{xc\_element} with the help of \textbf{datatype} keyword from Isabelle, 
as shown in \hyperref[figure:5]{Figure 3.3}
\begin{figure}[h!]
    \Snippet{exact-cover-basics}
    \caption{Definition of the encapsulation type}
    \label{figure:5}
\end{figure}
The constructor \textbf{V, C} and \textbf{L} stands for the variables, clauses and literals. \textbf{V} 
encapsules the variable, while \textbf{C} encapsules the clauses. The definition of \textbf{L} is a bit different.
To locate the literals exactly, we have to encapsule both the literal and the clause. The reason is that literals are 
also located by clauses. If we discard the clauses, the same literal may exist in many clauses, 
whereas there is only one literal covered in our construction.


\subsection*{Non-deterministic Construction of a Sub-collection}
When constructing a cover $S'$, we have included $\{c_i, p_{ij}\}$, where $j$ is the smallest number 
such that $p_{ij}$ is true under the assignment $\sigma$. However, in our implementation, 
we did not use an integer to index the positions. For this reason, we cannot deterministically choose 
a suitable $p_{ij}$. The predicate \textbf{SOME} resolves this problem---\textbf{SOME} \textit{x. P x}
returns an arbitrary $x$ that satisfies the predicate $P$.
\begin{figure}[h!]
\Snippet{exact-cover-prf} 
\caption{Non-deterministic Construction using SOME}
\label{figure:7}
\end{figure} 
Using this predicate, we are able to choose $\{c_i, p_{ij}\}$ non-deterministically, as defined in \textbf{constr\_cover\_clause} in \hyperref[fig:7]{Figure 3.4}. 
Another benefit of this approach is that we can also include $\{p_{ij}\}$ simultaneously. Thus, we are exempted from introducing another function 
to compute this sub-collection. Apart from the definition, the lemma \textbf{constr\_cover\_clause\_unfold} is also proven to remove the \textbf{SOME}
predicate and obtain the property of the encapsuled sub-collections. \\\\
The rest of the proof of the correctness follows the mathematical proof precisely. 
It involves showing the disjointness and covering of each part of the construction. 
Though it is bit lengthy, the general idea is clear and straightforward.

\subsubsection{Polynomial-time Complexity}
In the complexity analysis, it is necessary to determine the metrics on which the complexity is dependent. For 
a logical formula $F$, we will iterate all variables, clauses and positions. Hence all of them are needed as metrics. 
Nevertheless, the NREST implementation does not support a complexity bound with different metrics. 
As a result, we choose the maximum of all metrics and use it as our sole metrics\footnote{A few previous reductions in Karp21 project use the number of clauses 
as a metric. Since those reductions iterated only the clauses, this choice of metric is not an issue.}. Similarly, we define $max |X| |S|$ as the metrics 
for the instance of exact cover  $(X, S)$. Then we can define the NREST algorithm in \hyperref[fig:3.5]{Figure 3.5}.
\begin{figure}[!h]
    \centering
    \Snippet{exact-cover-poly}
    \caption{The NREST version of the reduction, SAT to XC}
    \label{fig:3.5}
\end{figure}
Let $n$ be the metric for the Satisfiability problem.
A comprehensive list of the complexity of each operation is given in \hyperref[table:3.1]{Table 3.1}. 
Since we follow the principles as stated in \hyperref[para1]{Chapter 2}, all complexity are easy to follow 
except for \textbf{mop\_literals\_of\_sat}, \textbf{mop\_var\_true\_literals} and \textbf{mop\_var\_false\_literals}. 
In all cases, we have to iterate the set of clauses, even though it is not necessary in the pen-and-paper reduction, 
because we defined the literals with the help of the clauses instead of the indices. Consequently, the resulting 
complexity is cubic instead of quadratic as stated in the reduction details. However, it is still 
a polynomial class, which is an acceptable result.
\begin{table}[!h]
    \centering 
    \begin{tabular}{| c | c | c |}
        \hline 
        Operation & Functionality & Complexity \\ 
        \hhline{|=|=|=|}
        mop\_vars\_of\_sat & collecting variables & $3n$ \\ 
        \hline 
        mop\_clauses\_of\_sat & collecting clauses & $3n$ \\ 
        \hline 
        mop\_literals\_of\_sat & collecting literals & $4n$ \\ 
        \hline 
        mop\_literal\_sets & encapsuling literals & $3n$ \\
        \hline 
        mop\_clauses\_with\_literals & non-deterministic construction & $6n$ \\ 
        \hline 
        mop\_var\_true\_literals & obtaining variables with negative existences & $3n^3$ \\ 
        \hline
        mop\_var\_false\_literals & obtaining variables with positive existences & $3n^3$ \\ 
        \hline
    \end{tabular}
    \caption{Complexity of operations in reduction SAT to XC.}
    \label{table:3.1}
\end{table}
\\\\
The main part of the proof of the polynomial-time complexity is automated following 
the style of previous works from the Karp21 project. 
Nevertheless, automation failed 
when it came to show the upperbound of the cardinality of the constructed sets, such as 
\textbf{card\_comp\_S} in \hyperref[fig:3.6]{Figure 3.6}. The function \textbf{comp\_S}
constructs the collection $S$ in the instance of exact cover. This lemma shows that 
the cardinality of the collection $S$ is above bounded by the quardruple of our metric.
\\\\
We analyzed the proof and found out the reason to be cardinality function. 
Whenever we want to show the (in)equality of cardinality between two sets, there are a few assumptions 
for the usable lemmas.
In most cases, these premises need to be manually passed to the lemmas. One frequently used lemma 
with such property is \textbf{card\_image} in \hyperref[fig:3.6]{Figure 3.6}. It requires to find 
an injective function between the preimage and the image. 
Without passing the premises to this lemma, we cannot obtain an inequality relationship.
\begin{figure}
    \centering
    \Snippet{exact-cover-polyaux}
    \caption{Details in the proof of the polynomial-time complexity, SAT to XC}
    \label{fig:3.6}
\end{figure}

\subsection{Example of the reduction}
In this final part, we present an example with a bried explanation how the reduction is correct for 
this certain example.\\\\
\textbf{Input:} A logical formula in conjunctive normal form 
\begin{align*}
    F := (x_1 \lor x_2 \lor \neg x_3) \land (\neg x_1 \lor x_2) \land (x_1 \lor x_3)
\end{align*}
\textbf{Output:}
The constructed set is
\begin{align*}
    X := & \{ x_1 ,x_2, x_3\} \cup \{c_1, c_2, c_3\} \cup 
    \{p_{11}, p_{12}, p_{13}, p_{21}, p_{22}, p_{31}, p_{32}\}
\end{align*}
The constructed collection is 
\begin{align*}
    S : = & \{\{p_{11}\}, \{p_{12}\}, \{p_{13}\}, \{p_{21}\}, \{p_{22}\}, \{p_{31}\}, \{p_{32}\}\} \\
      & \cup  \{\{c_1, p_{11}\}, \{c_1, p_{12}\}, \{c_1, p_{13}\}, \{c_2, p_{21}\}, 
      \{c_3, p_{31}\}, \{c_3, p_{32}\}\} \\ 
      & \cup \{\{x_1, p_{11} p_{31}\}, \{x_1, p_{21}\}, \{x_2, p_{12}, p_{22}\}, \{x_2\},
      \{x_3, p_{32}\}, \{x_3, p_{13}\}\}
\end{align*}
\textbf{Validity:} Apparently, the only valid assignment $\sigma$ of $F$ is given by 
\begin{align*}
    \sigma = \{x_1 = \bot, x_2 = \top, x_3 = \top \}
\end{align*}
We construct an exact cover $S'$ by 
\begin{align*}
    S' = \{\{c_1, p_{22}\}, \{c_2, p_{21}\}, \{c_3, p_{32}\},
    \{x_1, p_{11}, p_{31}\}, \{x_2\}, \{x_3, p_{31}\},
    \{p_{22}\}\}
\end{align*}

\section{Exact Hitting Set}
The hitting set problems are variants of the set covering problems. Essentially, they are two different ways 
of viewing the same problem. Just as the Hitting Set\footnote{Note that the exact hitting set problem was referred to as the hitting set problem in Karp's work, whereas 
it is generalized to be another problem nowadays.} is a variant of Set Cover, the Exact Hitting Set is 
a variant of Exact Cover. 
\problem{Exact Hitting Set}{A collection of sets $S$}{Is there a finite set $W$ s.t. the intersection of $W$ and 
each element $s \in S$ contains exactly one element?}
\begin{myalign}
   \textbf{Exact Hitting Set} := \{S\ |\ \exists W.\ \forall s \in S.\ |W \cap s| = 1\}
\end{myalign}
We call $W$ to hit $S$ exactly.
 

\subsection{Reduction Details}
Given an instance of Exact Cover $(X, S)$, the instance of Exact Hitting Set $C$ is constructed by 
\begin{align*}
    C = \{\{s | u \in s, s \in S \} | u \in X\}
\end{align*}
Thus, $C$ is the set of sub-collections denoted by $c_u = \{s | u \in s, s \in S\}$. All sets in $c_u$ share the same element $u$.
\begin{definition}[Reduction XC to XHS]
    Given an instance of Exact Cover $F$, 
    it is reduced to an instance of Exact Hitting Set $C$ as presented above.
\end{definition}
It is possible to show the correctness of the reduction with this defintiion.
\begin{lemma}[Soundess]
    Let $(X, S)$ be an instance of Exact Cover. A collection $C$ reduced from $(X, S)$ is then an instance of the exact hitting set.
\end{lemma}
\begin{proof}
For the soundness of the reduction, it suffices to show 
\begin{align*}
    \exists W.\ \forall c_u \in C.\ |W \cap c_u| = 1
\end{align*}
Let $W = S'$, where $S'$ covers $X$ exactly. Moreover, let $c_u$ be an arbitrary element of $C$. 
Since $S'$ covers $X$ exactly, there is exactly one $s \in S'$ that contains $u$. 
$c_u$ is a collection of sets in $S$ that contain $u$.
Thus, $s$ is the only element in the set $S' \cap c_u$. 
Consequently, $S'$ hits $C$ exactly.
\end{proof}
Then, we show the completeness.
\begin{lemma}[Completeness]
    Let the collection $C$ be a collection reduced from a pair $(X, S)$. If $C$ is an instance of the exact hitting set, 
    $(X, S)$ has to be an instance of Exact Cover.
\end{lemma}
\begin{proof}
The proof of the completeness shares a similar construction. 
The only difference is that $W$ is not necessarily a subset of $S$.
Nevertheless, there exists a subset $W' \subseteq W$ s.t. it is not only a subset of $S$, but it
also satisfies the property $\forall c_u \in C. |W' \cap c_u| = 1$. \\\\
Let $S' = W'$. For each $u \in X$, there is exactly one set $s \in W'$ s.t. $u \in s$. 
This ensures that $X$ is fully covered and each element in $X$ is existent in only one set.
Moreover, $W'$ is a subset of $S$, indicating that all sets in $W$ only contain elements of $X$. 
Hence $W'$ is also disjoint.  
\end{proof}
With the correctness proven, we show that the reduction is in polynomial time. 
\begin{lemma}[Polynomial Complexity]
    The construction of $C$ from $(X, S)$ can be computed within polynomial time. 
\end{lemma}
\begin{proof}
    In our reduction, it is necessary to iterate the set $X$ and the collection $S$ in a nested loop. 
With the cardinality $|X|$ and $|S|$ as the metrics, it is obvious that the reduction costs the complexity of $\bigO{|X||S|}$.
\end{proof}
In the end, we summarize the lemmas to obtain the main theorem.
\begin{theorem}
    The presented reduction is a polynomial-time reduction from Exact Cover to Exact Hitting Set.
\end{theorem}

\begin{figure}[!ht]
    \Snippet{exact-hitting-set-def}
    \Snippet{exact-hitting-set-reduction}
    \caption{Definition of the reduction, XC to XHS}
    \label{fig:8}
\end{figure}

\subsection{Implementation Details}
\subsubsection{Definition of the reduction}
Since the Exact Cover problem is defined over a finite set $X$ and a finite collection $S$, we have to check if the $X$ and $S$ are finite and 
if $S$ is a collection of $X$. Thus, a condition statement which checks this requirement is added to the implementation in \hyperref[fig:8]{Figure 3.7}
\subsubsection{Polynomial-time Complexity}
\begin{figure}[!ht]
    \Snippet{exact-hitting-set-poly}
    \caption{The NREST version of the reduction, XC to XHS}
    \label{fig:3.9}
\end{figure}
\begin{table}[!ht]
    \centering 
    \begin{tabular}{| c | c | c |}
        \hline 
        Operation & Functionality & Complexity \\ 
        \hhline{|=|=|=|}
        mop\_check\_finiteness\_and\_is\_collection & checking the requirement & $1$ \\ 
        \hline 
        mop\_construct\_sets & constructing the new instance & $3n^2$ \\
        \hline
    \end{tabular}
    \caption{Complexity of operations in reduction XC to XHS.}
    \label{table:3.2}
\end{table}
We determine the size of the instance of Exact Hitting Set $C$ as $|C|$. According to the paradigm, 
we define the NREST algorithm in \hyperref[fig:3.9]{Figure 3.9}. In the same way as the
implementation for the reduction from Satisfiability to Exact Cover, we can only use one single metric. 
Let $n = max |X| |S|$ be the metric for Exact cover. 
The complexity of each operation is given in \hyperref[table:3.2]{Table 3.2}. 
Just as concluded in the pen-and-paper proof, the reduction costs quadratic time.
\begin{figure}[!h]
    \Snippet{exact-hitting-set-aux}
    \caption{Details in the proof of the polynomial-time complexity, XC to XHS}
    \label{fig:3.10}
\end{figure}\\\\
The proof of the polynomial complexity is mostly automated after unfolding the necessary definitions. 
However, an additional step is required for indicating the relationship between the sizing functions. 
While it holds $|C| = |X|$, 
the size of the Exact Cover problem is defined by $max |X| |S|$ instead of $|X|$. Hence we can
only conclude that the size of the Exact Hitting Set is less equal than the size of the Exact Chapterover. 
The proof automation will then fails in showing $|C| \leq |S| \cdot |S| + 1$ when $|S| \geq |X|$. 
For this reason, we have to prove one additional lemma about the cardinality of the 
Exact Hitting Set as in \hyperref[fig:3.10]{Figure 3.9}.





