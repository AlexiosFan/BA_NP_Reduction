\newcommand{\abstractname}{Abstract}


\chapter{\abstractname}

\section*{English}
NP-hardness is a fundamental concept in the complexity theory. It represents a class of problems that 
are hard to be computed by a polynomial-time algorithm. Polynomial-time reductions are used 
to classify the NP-hardness of such problems. While proofs of the correctness of the polynomial-time reductions 
were limited to pen-and-paper proofs, 
it is now possible to reproduce and verify these proofs with the aid of computers. 
There has been an on-going effort to formalise and verify the classical NP-hard problems, 
domonstrating capability of the interactive theorem prover Isabelle in verifying NP-hardness.
On the basis of this effort, we continue to verify the NP-hardness of six problems, 
including Exact Cover, Exact Hitting Set, Subset Sum, Partition, Knapsack, and Zero-One Integer Programming.

\section*{Deutsch}
NP-Schwere ist ein grundlegendes Konzept in der Komplexitätstheorie. 
Sie steht für eine Klasse von Problemen, die mit einem Polynomialzeit-Algorithmus schwer zu berechnen sind. 
Polynomielle Reduktionen werden verwendet, um die NP-Schwere solcher Probleme zu klassifizieren. 
Während Beweise für die Korrektheit der polynomiellen Reduktionen früher auf dem Papier beschränkt waren, 
ist es nun möglich, die Beweise mithilfe von Computern zu reproduzieren und zu verifizieren. 
Es gibt laufende Bemühungen, die klassischen NP-schweren Probleme zu formalisieren und zu verifizieren. 
Dabei wurde die Fähigkeit des interaktiven Theorembeweisers Isabelle zur Verifizierung der NP-Schwere nachgewiesen. 
Basierend darauf verifizieren wir weiterhin die NP-Schwere von sechs Problemen, 
einschließlich Exact Cover, Exact Hitting Set, Subset Sum, Partition, Knapsack und Zero-One Integer Programming.