\newcommand{\abstractname}{Abstract}


\chapter{\abstractname}
NP-hardness is a fundamental concept in the complexity theory. It represents a class of problems that 
are hard to be computed by a polynomial-time algorithm. Polynomial-time reductions are used 
to classify the NP-hardness of such problems. While proofs of the correctness of the polynomial-time reductions 
were limited to pen-and-paper proofs, 
it is now possible to reproduce and verify these proofs with the aid of computers. 
There has been an on-going effort to formalise and verify the classical NP-hard problems, 
domonstrating capability of the interactive theorem prover Isabelle in verifying NP-hardness.
On the basis of this effort, we continue to verify the NP-hardness of six problems, 
including Exact Cover, Exact Hitting Set, Subset Sum, Partition, Knapsack, and Zero-One Integer Programming.